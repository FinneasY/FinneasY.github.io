\documentclass[12pt]{article}
\usepackage[utf8]{inputenc}
\usepackage{ctex}
\usepackage{amsmath, amssymb}
\usepackage[margin=1in]{geometry}

\title{代际心理问题传递对经济增长的影响:一个内生增长模型}
\author{}
\date{}

\begin{document}

\maketitle

\section*{模型假设}

将个体生命周期分为两个阶段,分别是青年阶段、中年阶段和老年阶段,假设个体每一期都有一单位的时间禀赋。在青年阶段,个体的心理状态受亲代影响,此时亲代会无偿为个体提供生活消费成本、教育成本和心理成本,同时政府也会对个体的心理问题包括教育提供财政补助。个体在中年阶段进行工作和消费,同时要拿出一定收入用于解决子女心理健康问题,还要提供对子女的教育支持以及基本的生活消费支出。个体在老年阶段退休并消费中年阶段的储蓄。在个体同质性假设的基础上,本文假设不存在资本资本借贷市场,即个体不能从借贷市场上获得心理问题的补偿和教育资金,只能从父母那里获取。亲代和子代彼此遵守利他原则,父代对子代进行无偿的抚养。同时,假设每代人的收入足够其医生消费,且其消费不会超过其一生的总收入。

根据跨期迭代模型的基本思路,福利水平定义为个体在中老年时期消费。同时假设个体的效用水平与子女的收入水平相关。基于上述假设,个人效用函数为:
\[
U^i_t = lnC^i_{1t} + \beta_{t+1}C^i_{2t}+ \alpha lnY^i_{t+1}
\]
其中,$U^i_t$ 表示个体 $i$ 的终身效用,$C^i_{1t}$ 、$C^i_{2t}$ 分别表示个体 $i$ 在青年、中年时期的消费,$Y^i_{t+1}$ 表示($t+1$)时期出生个体$i$的工资收入。

\subsection*{1. 模型设定}
个体在青年阶段的心理问题受到亲代心理问题的影响,同时在此阶段接受教育,假定个体人力资本水平与亲代人力资本、亲代心理问题、教育水平和政府部门的财政支出相关,其决定公式如下:
\[
h^i_{t+1}=B(h^i_t)(m^i_t)(s^i_{t+1})(g^i_{St+1})(g^i_{Mt+1})^{}
\]
其中,$h^i_{t+1}$ 表示 $t+1$ 时期出生的个体 $i$获得的人力资本,$h^i_t$ 表示 $t$ 时期出生的个体 $i$的人力资本水平,$m^i_t$ 表示 $t$ 时期出生的个体 $i$ 的心理问题水平,$s^i_t$ 表示个体在 $t+1$ 时期获得的亲代教育投资,$g^i_{St+1}$ 表示政府部门对 $t+1$ 时期出生的个体 $i$ 的教育支出,$g^i_{Mt+1}$ 表示政府部门对 $t+1$ 时期出生的个体 $i$ 的心理问题支出。$B>0$ 表示个体人力资本积累的技术参数,







我们构建一个具有内生人力资本积累的重叠世代模型(OLG),借鉴 Romer《高级宏观经济学》第3章的内生增长模型,并结合 Abramson 等人(2024)提出的心理健康动态机制。

\subsection*{1. 个体生命周期与心理健康状态}

个体活两期:青年期投资人力资本,成年期进行生产。每个个体具有一个连续的心理健康状态 $m_t \in [0,2]$,其中数值越大表示心理问题越严重。

心理状态影响个体对未来的主观预期,通过函数 $\kappa(m_t)$ 表示“负面预期强度”,满足:
\[
\kappa'(m_t) > 0,\quad \kappa(0) = 0
\]
即心理状态越差,负面预期越强,理性回报的低估程度越大,进一步抑制投资行为。

\subsection*{2. 人力资本投资与最优化}

青年期选择投资水平 $e_t \in [0,1]$,成本为 $c(e_t) = \frac{1}{2} e_t^2$,形成的人力资本为:
\[
h_{t+1} = (1 - \kappa(m_t)) \cdot e_t
\]
成年期产出为 $y_{t+1} = A h_{t+1}$,其中 $A > 0$ 为技术水平。

最大化效用函数:
\[
U_t = (1 - e_t) + \beta A (1 - \kappa(m_t)) e_t
\]

一阶条件推导:
\[
\frac{dU_t}{de_t} = -1 + \beta A (1 - \kappa(m_t)) = 0 \Rightarrow e_t^* = \min \left\{1, \beta A (1 - \kappa(m_t)) \right\}
\]

\subsection*{3. 代际心理状态传递}

本文假设心理状态为连续变量 $m_t \in [0,2]$,其跨代传递机制为:
\[
m_t = \gamma m_{t-1} + \epsilon_t, \quad \gamma \in [0,1)
\]
其中 $\gamma$ 表示代际心理传递的强度,$\epsilon_t$ 是代表外生干预(如政策或治疗效果)的冲击项。$\gamma$ 越大,表示心理问题越容易在代际中“遗传”。

\textbf{负面预期函数}保持为 $\kappa(m_t)$,并满足 $\kappa'(m_t) > 0$,即心理状态越差,负面预期越强。


\subsection*{4. 总体产出与增长率}

由个体的最优投资水平 $e_t = \beta A (1 - \kappa(m_t))$ 代入人力资本形成公式 $h_{t+1} = (1 - \kappa(m_t)) e_t$,得到:

\[
h_{t+1} = \beta A (1 - \kappa(m_t))^2
\]

总产出为所有个体人力资本之和乘以技术水平 $A$,即:
\[
Y_{t+1} = A \cdot \mathbb{E}[h_{t+1}] = A \cdot \mathbb{E}[\beta A (1 - \kappa(m_t))^2] = \beta A^2 \cdot \mathbb{E}[(1 - \kappa(m_t))^2]
\]

总人力资本为:
\[
H_{t+1} = \int e_t^i \cdot (1 - \kappa(m_t^i)) \, di
\]
总产出为 $Y_{t+1} = A H_{t+1}$,增长率为:
\[
g_{t+1} = \frac{Y_{t+1}}{Y_t} - 1 = \frac{H_{t+1}}{H_t} - 1
\]

\subsection*{5. 关于产出评估的讨论}

\textit{作者注:} “我不确定是否可以评估产出,这只是我随便的一个意见,你可以拒绝我的意见也可以不拒绝。”

\textbf{回应:} 尽管我们没有引入物质资本或技术变迁,但由于生产函数是 $Y = A H$ 的线性形式,因此 $Y_t$ 直接由 $H_t$ 决定,故心理健康传递对经济增长的影响是可以被评估和量化的。

\subsection*{6. 心理状态 $m_t$ 与代际传递强度 $\gamma$ 对产出 $Y_{t+1}$ 的影响:显式推导}

总产出为:
\[
Y_{t+1} = \beta A^2 \cdot \mathbb{E}\left[(1 - \kappa(m_t))^2\right]
\]
其中 $m_t = \gamma m_{t-1} + \epsilon_t$,$\epsilon_t$ 为独立冲击。

对 $\gamma$ 求导,得到:
\[
\frac{dY_{t+1}}{d\gamma} = -2 \beta A^2 \cdot \mathbb{E}\left[
(1 - \kappa(m_t)) \cdot \kappa'(m_t) \cdot m_{t-1}
\right]
\]

为简化分析,我们设负面预期函数为线性形式 $\kappa(m) = \theta m$,其中 $\theta \in (0, 1)$ 表示心理状态对负面预期的敏感程度。

\paragraph{线性函数特例:}
若设 $\kappa(m) = \theta m$,则有:
\[
\frac{dY_{t+1}}{d\gamma}
= -2 \beta A^2 \theta \cdot \mathbb{E}[(1 - \theta m_t) \cdot m_{t-1}]
\]

由于该导数项为:
\[
\frac{dY_{t+1}}{d\gamma}
= -2 \beta A^2 \theta \cdot \mathbb{E}[(1 - \theta m_t) \cdot m_{t-1}]
\]
我们希望其为负(表示代际心理传递强度越大,产出越低),因此必须满足:
\[
\mathbb{E}[(1 - \theta m_t) \cdot m_{t-1}] > 0
\]
一种保证这一项为正的方式是确保心理状态 $m_t$ 不太大,即 $\theta m_t < 1$ 成立,从而 $(1 - \theta m_t) > 0$。因此,为了确保该结论严格成立,我们建议将 $\theta$ 的上界设置为一个中等偏小的值(如 $\theta < 0.5$),以增强模型稳定性并保持负向传导结果。

上式中各参数含义如下:
\begin{itemize}
  \item $\beta$:代表个体对未来的贴现因子;
  \item $A$:技术常数,表示单位人力资本的产出效率;
  \item $\theta$:心理状态对负面预期的敏感性,取值 $\theta \in (0,1)$;
  \item $m_t$:子代心理状态,受上一代影响;
  \item $m_{t-1}$:父代心理状态;
  \item $\gamma$:代际心理传递强度;
\end{itemize}

\textbf{经济解释:} 该公式说明,当心理状态的负面预期函数呈线性时,产出对代际心理传递强度 $\gamma$ 的变化率与上一代心理问题程度 $m_{t-1}$ 成正比。若 $\gamma$ 越大,即心理问题更容易传递,则整体社会平均产出 $Y_{t+1}$ 会因为子代的负面预期增强而下降。

\paragraph{结论:}
代际心理传递强度越高($\gamma$ 越大),若上一代心理问题越严重($m_{t-1}$ 趋大),会显著抑制后代人力资本积累,从而降低产出。
\end{document}
